\documentclass[10pt]{article}
\usepackage{url}

\def\eg {{\it e.g.}, }
\def\ie {{\it i.e.}, }

\setlength{\topmargin}{-1in}
%\setlength{\headsep}{0in}
%\setlength{\headheight}{0in}
\setlength{\textheight}{9.5in}
%
\setlength{\oddsidemargin}{0in}
\setlength{\evensidemargin}{0in}
\setlength{\textwidth}{6.5in}
%
%\setlength{\itemsep} {0pt}
%\setlength{\parsep} {0pt}
%\setlength{\partopsep}{0pt}
%\setlength{\topsep}{0pt}

\pagestyle{empty}

\begin{document}
\thispagestyle{empty}

%\vspace{-.5in}
\title{CSCI 13500\S 01 -  Software Analysis \& Design 1 \\ Spring 2017}
\date{}
\author{Raffi Khatchadourian}
\maketitle

\thispagestyle{empty}
%\begin{itemize}
\paragraph*{\bf Meeting Time:} Tuesday and Thursday, 5:35-6:50 PM, N-1036

\paragraph*{\bf Instructors:}\ \\
  \begin{tabular}{ll}
    Eric Schweitzer & Raffi Khatchadourian \\
    eric.schweitzer@hunter.cuny.edu & raffi.khatchadourian@hunter.cuny.edu\\
    212-772-4349 & 212-772-5213\\
    N-1000E & N-1000C\\
    Wednesday and Friday, 3:30 to 5:00 & Tuesday and Thursday, 4:30 to 5:30\\
    or by appointment & or by appointment
  \end{tabular}\\
  Send plain (ASCII) text email, from your ``myhunter'' account. HTML, MS-Word docs and email from other sources are likely to be ignored. 

\paragraph*{\bf Pre-/Co-requisites:} 
  The prerequisite is CSCI 127 or instructor's permission. 
  At the very least, you should have written, compiled, and run a
  $>$1 page program containing iteration (while/for) and selection
  (if) statements. 

  CSCI 136 is a co-requisite, and you will not receive a grade in 135 
  if you don't take 136.
  If you have already passed 136, you {\bf must} contact the instructors before the
  2nd class, so that you can be added to a bb section and your work
  can be graded.  

\paragraph*{\bf Questions:} %Raffi: I'm not sure I like this ... we should be ``accessible''
  Your \emph{primary} contact person is your 136 instructor, however, you may also contact your 135 instructors.
  The beginning of each CSCI 136 class will be devoted to answering
  questions, and you may also use the 135 blackboard forum. 

\paragraph*{\bf Bulletin Board:}
  You should check the Blackboard (\url{http://bb.hunter.cuny.edu})
  sites for both 135 and 136 regularly, since all class material will
  be posted there.
  Please make sure you have configured bb to use your preferred email
  address (your Hunter email address, by default), since you are
  responsible for any email the instructors might send there. 

\paragraph*{\bf Text:}
  Walter Savitch, Absolute C++, 6th Edition, ISBN 0-13-397078-7

  You are responsible for all the material on the reading list whether
  or not covered in lectures.  

\paragraph*{\bf Grading:} 
  \begin{itemize}
  \item[25\%] Programming Projects (3-5)
    This may not seem like a lot, but beware that many test questions
    will be based directly on these programs.  
  \item[75\%] Tests (3, equally weighted). 
    Test 3 will be according to Hunter's official final exam
    schedule for this class. This is currently scheduled for Tuesday, May 23,
    5:20-7:20, but that is subject to change by the College. 
  \end{itemize}
  Final grades will be assigned based on the above and the following
  guidelines:\\
%  \begin{itemize} %Raffi: I think this should be replaced by numerical values
% Eric: Do you mean like "90--100: You understand the concepts ..."?
%  \item[A:] You understand the concepts well enough to successfully
%    understand and solve a problem by implementing a C++ program on
%    your own. 
%  \item[B:] Between the above and below.
%  \item[C:] You understand all basic C++ concepts, but have trouble
%    implementing a complete correct program. 
%  \item[F:] You do not understand basic C++ concepts, or are unable to
%    implement programs.
  %  \end{itemize}
$
  \begin{array}{rclcl}
    average&&grade&&average\\
    97.5 & \leq & A+ &  &  \\
    92.5 & \leq & A & < & 97.5 \\
    90 & \leq & A- & < & 92.5\\
    87.5 & \leq & B+ & < & 90 \\
    82.5 & \leq & B & < & 87.5 \\
    80 & \leq & B- & < & 82.5 \\
    77.5 & \leq & C+ & < & 80 \\
    70 & \leq & C & < & 77.5 \\
    65 & \leq & D & < & 70 \\
       &      & F & < & 65
   \end{array}\\$

\paragraph*{\bf Late Policy:}
  Late penalties for programming projects are: 5\% within 2 hours of
  the due date/time, 10\% within 24 hours, and an additional 10\% for
  each day after that. 
  No homework is accepted after 1 week. 

  Make-up tests are not given under any circumstances.
  If there are extraordinary circumstances why you are unable to take 
  a test, you must provide documentation {\it prior} to the test to
  avoid getting a zero.

\paragraph*{\bf Software:}
  The standard Linux/Unix/MacOS C++ compiler is g++. 
  There are several free Linux OSs which you can install for
  dual-booting with Windows, including \url{http://www.ubuntu.com}. 
  If you want a Linux-style environment for windows that doesn't
  require installing Linux, \url{http://cygwin.com} is an
  alternative. You will be able to ssh into ``CSnet'' to compile and run
  your programs in our Linux environment. The G-Lab machines are the arbiter of
  ``compile and run''.

\paragraph*{\bf Course Goals:}
This course is: 
\begin{itemize}
\item
  An introduction to software development, using the C++ programming
  language. 
  Software development is a skill involving understanding the problem
  being modeled, as well as expressing a solution using a programming
  language.
  Thus, it has both conceptual and technical components.
  The successful student will be able to clearly and logically
  transform a problem, while being comfortable with C++ to express the
  transformed problem. 
\item
  A preparation for further courses in computer science.
  This course comprises the ABCs of computer science, and you're not
  allowed to forget it anymore than you can forget the alphabet after
  kindergarten.
  Students who expect to take more advanced courses in computer
  science need to go beyond ``understanding'' the material presented
  in this course -- they need to master it. 
\item
  Time consuming; very, very time consuming.  
  Any programming course takes up a lot of a student's time. 
  In addition to the time spent in class, most students will need to
  spend between 10 and 20 hours a week at a computer. 
  That makes for a total of 15-25 hours a week -- that's 15-25 hours a
  week you must dedicate JUST FOR THIS COURSE, no kidding!
\end{itemize}
This course is NOT:
\begin{itemize}
\item 
  An introduction to computers in general. we will not cover: Linux,
  networks, databases, etc. 
  This course teaches a specialized skill -- programming -- and only
  programming. 
  You should already possess basic computer skills such as compiling
  simple computer programs, editing files, manipulating files, etc. 
\item
  An overview of the C++ language. 
  C++ is a huge language with a lot of highly technical details. 
  We will cover the fundamentals of C++, but the focus is on designing
  algorithms and solving problems.

\item
  A good idea to take if you are working full time and taking a full
  course load, or, for any other reason(s) you don't have a lot of
  free time to devote to CSCI 135.  Although much of the material
  isn't especially difficult, it usually requires many, many hours to
  master.  It is difficult to understand software development concepts
  without sitting in front of a computer many hours a week actually
  writing and debugging programs.  Be honest with yourself. Make sure
  this course is for you, -- now -- at this point in your academic
  life. If you would like to discuss the time requirements further
  please feel free to come talk with the instructors.
\end{itemize}

This course provides a 'first step' towards the
following department's {\bf learning goals}:
\begin{itemize}
\item
  Have a deep practical knowledge of one widely used programming
  language
\item
  Be experienced in working in at least two widely used operating
  system environments
\item
  Be able to apply principles of design and analysis in creating
  substantial programs and have experience working in teams on
  projects of moderately realistic scope.
\end{itemize}

%\pagebreak
\paragraph*{\bf How to Learn:}
\begin{itemize}
\item

  From the beginning, students will be expected to work independently 
  outside of the lectures. 
  There will be very little ``hand holding'' in the course -- you are 
  expected to find your way around your computer on your own . 
  For example, the way each of you will save your work, compile/debug
  an assignment, etc. will vary. 
  These techniques will not be covered in class. 
  Get started NOW
  (especially if you are going to install your own compiler and/or OS
  software). 
  The first programming project is due soon.
\item
  There will be many obstacles to overcome, both in absorbing the many 
  examples of C++ programs and in doing the assignments. 
  Attacking the obstacles head on, outside of class time, in front of
  a computer, is the key to success.  
  Keep trying, if your program is not working, try again.  
  Still not working? Try again.  And again.  And again. 
  When it comes to programming, the learning is in the doing. 
  There is no substitute for spending many hours in front of a
  computer -- trying and failing, trying and failing, trying and
  failing -- until you finally get your program up and running
  correctly. 
  Every time you fail, you actually learn quite a bit, and to pass the
  course, you will repeat this trying-failing cycle many times, every
  week of every month during the entire semester. 
  There are virtually no ``slow points'' during the semester.
\item
  There will be approximately four programming projects. 
  By far, the main cause for an unsatisfactory final grade is falling
  behind on the assignments. 
  Exams are largely based on the programming assignments. 
  If you don't do the assignments, on time, you will almost certainly
  not pass the tests.
 \end{itemize}

\paragraph*{\bf Other}:
We ask that all cell phones, pagers, etc. be silent in class. Any
electronics in use should be used for class related activities.
Violations of these ``requests'' will result in the loss of
2 points from you final average (per occurrence).

  All course material (including lectures, solutions, etc.) is owned
  by the instructor and protected by {\bf copyright}.
  You may use the material for yourself, but any other use (including
  posting on websites, whether free or not) is illegal without our
  express written permission.

  We take {\bf academic honesty} very seriously, and any violation
  results in  an automatic F for the course along with sanctions in
  accordance with Hunter College procedure. 

   Hunter College regards acts of academic dishonesty
  (\eg plagiarism, cheating on examinations, obtaining unfair
  advantage, and falsification of records and official documents) as
  serious offenses against the values of intellectual honesty.
  The College is committed to enforcing the CUNY Policy on Academic
  Integrity and will pursue cases of academic dishonesty according to
  the Hunter College Academic Integrity Procedures.

  In compliance with the {\bf American Disability Act}
  of 1990 (ADA) and with Section 504 of the Rehabilitation Act of
  1973, Hunter   College is committed to ensuring educational parity
  and accommodations for all students with documented disabilities
  and/or medical conditions.
  It is recommended that all students with documented disabilities
  (Emotional, Medical, Physical and/ or Learning) consult the Office
  of AccessABILITY located in Room E1124 to secure necessary academic
  accommodations.
  For further information and assistance please call
  (212-772-4857)/TTY (212-650-3230).



  This syllabus is subject to change should the need arise.
\end{document}

